\documentclass[,a49pt]{book}



\begin{document}


\begin{flushleft}
\textbf{88}\hspace*{1cm} \texttt{CHAPTER SEVEN}
\end{flushleft}

\vspace*{0.5cm}
steps are challenging when using asynchronous text-based software to communicate, often making effective Net-based interviews more difficult to conduct than face-to-face interviews. However, this challenge is offset when travel expenses and the need to schedule with subjects are eliminated, resulting in significant saving of cost and time.\\
\hspace*{0.5cm} As in any form of interaction, familiarity with the participant's culture, values, and use of language greatly assists communication. Communicating in a text-based asynchronous environment is difficult enough in one's own language and culture. This difficulty is magnified is magnified when the e-researcher is not familiar with the participant's lan-guage (e.g., lexicon, syntax, phonemes) and culture.

If a faux pas occurs as a result of inappropriate language use or lack of understanding of the culture, it may lead to mis-understanding between the participant and e-researcher and result in an inability to draw out relevant information. The e-researcher cannot use the body language, facial expressions, poignant pauses, or other paralinguistic clues provided during traditional interviews as cues to the subjects' deeper feelings. Subjects who are familiar with text-based communication may develop skills and techniques to overcome this limitation through the use of emoticons and other means of communicating affect in their mes-sages, but many potential subjects may not have these skills.\\

\hspace*{0.5cm} Gaining trust is also essential to successful data collection in interviews. Even when trust is initially gained, it is often fragile, and efforts must be made to build on and preserve it. As the aim of the interview is to acquire an understanding of the par-ticipant's perspective through open and honest dialogue, it is important not only to establish but also to maintain trust throughout the interview experience. When con-ducting face-to-face interviews, establishing trust can take days, weeks, or even months of time in the environment, observing and often making only painstakingly small advancements in an attempt to integrate. How can the e-researcher achieve rapport and trust over the Internet?\\

\hspace*{0.5cm} The degree of trust necessary for successful Net-based interviews depends on the attitudes and skills of the participants and the sensitivity of the interview subject. The participant's attitude toward and knowledge of Internet technologies are critical to a successful Net-based interview.If participants have a bias against Net technologies, the medium will negatively influence their responses. According to Fontana and Frey (1994), the type of interview selected, the techniques used, and way in which the data is collected will ''all come to bear on the results of the study'' (p.370).\\

\hspace*{0.5cm} Ideally, participants selected for Net-based interviews should have considerable prior experience and a high degree of comfort in communicating using the Net. To reduce the challenges associated with learning new software, we suggest the use of per-sonal email (as opposed to threaded discussion groups or other collaborative tools) to conduct the interview. This way, interview questions will appear as ordinary email from the researcher, and the techniques for responding, saving, or forwarding will already be familiar to most participants. If, however, the participant is new to asyn-chronous Internet communication technologies, or to a particular conferencing sys-tem, more time will be spent learning how to use the software than responding to the interview questions, which often lowers the insightfulness, quality, or usefulness of the communication.\\

\newpage
\begin{flushright}
 \texttt{SEMI-STRUCTURED AND UNSTRUCTURED INTERVIEWS} \hspace*{1cm} \textbf{89}
\end{flushright}

\vspace*{0.5cm}
\hspace*{0.5cm} The vulnerability that the participant may feel in sharing information also affects the ability of the interviewer to gain trust. For example, if the interview questions require the participant to share information that may be self-incriminating, the Net may not be an appropriate communication venue. Net-based communication is not as secure as the privacy of a personal face-to-face interview. Specifically, there are other people (e.g.,server maintenance personnel and/or hackers) who may have access to the interview transcripts as they are stored on a system server or as they are intercepted in transmission. Although most research institutions require their staff to sign a confi-dentiality form with respect to any information they may have access to, data may, nev-ertheless, be less secure than with other forms of interview collection, such as researcher notes.\\

 Assuring confidentiality and explaining the techniques used to protect the privacy of participants are important components of obtaining informed consent and building trust. Ways to provide informed consent, to identify the risks of participating in research on the Net, and to reduce the risks through the use of encrypted software were explained in Chapter 6.\\
 \hspace*{0.5cm} One step that you can take to determine if the Net is a suitable communication medium in which to conduct an interview is to spend time doing background work on the selected or potential participants. Much can be learned, for example, from reading personal Web pages or published works and/or asking people who the parti-cipants.\\

  Of particular importance is to collect information that indicates whether the participants are comfortable communicating with asynchronous tools.
If this informa-tion cannot be attained through these sources, you may want to consider conducting a brief, preliminary opinion survey about the Net. Creating and administering a survey can be done quickly using one of the commercial or free Web-based survey forms dis-cussed in Chapter 11 or by including questions in a short email to potential respon-dents, and can be an effective way to determine if negative biases exist. Such a survey could ask for simple yes/no or agree/disagree opinions on such questions as:\\

\vspace*{0.2cm}

\begin{itemize}
  \item How long have you used the Internet for communication (e.g., email, mailing list servers, Usenet groups, etc.)?\\
  \item Do you think the Internet is a useful form of communication?\\
  \item Do you use the Internet for personal communication?\\
  \item Do you think the Internet is a reliable communication medium?\\
  \item Are you comfortable with expressing your personal opinions through Internet communication platforms, such as email?\\
\end{itemize}

Should any of these questions be answered with a negative response, you might want to re-consider selecting the participant(s) for Net-based interviews.\\
\hspace*{0.5cm} Additionally, it is important that the e-researcher consider whether or not nonverbal and contextual elements are essential to gathering and interpreting the data. There are four kinds of nonverbal elements considered by some qualitative researchers to be essential to the interviewing process. These include \emph{proxemic, cbrone-mic, kinesic,} and \emph{paralinguistic} (Fontana \& frey, 1994). Proxemic communication refers to the interpersonal space used to communicate attitudes.

\newpage
\begin{flushleft}
\textbf{90}\hspace*{1cm} \texttt{CHAPTER SEVEN}
\end{flushleft}

\vspace*{0.5cm}
Chronemic communication refers to the use of speech patterns, such as pacing and the length of silence in conversation. Kinesic communication refers to body posture and movements. And paralinguistic communication refers to variations in pitch and qual-ity of voice. Are these nonverbal elements essential to the interviewing process or can they be ignored or substituted in text-based interaction? When using text-based asyn-chronous communication software, these nonverbal elements are not present--so we must ask ourselves, how will this influence the data collected? Obviously, different types of data collection from interviews will be more or less suitable to different types of research questions. For example, questions related to highly sensitive and personal subjects will be more dependent on paralinguistic clues for interpretation than ques-tions related to one's job or to experiences in public places (such as a classroom or a workplace setting). \\

\hspace*{0.5cm} Determining if the research question matches the data collection process requires the e-researcher to envision how these unique constraints of Net-based inter-views will change the way the data is gathered and interpreted. Determining what types of interviews to conduct using asynchronous text and discovering the appropri-ate participants for text-based interviews are challenges for e-researchers. Due to these limitations, Net-based interviews are used by some researchers as a filtering technique for initial interviews that are then followed by more extensive interviews conducted by telephone or face-to-face. \\

\hspace*{0.5cm} Another limitation of text-based asynchronous interviews is asking the partici-pants to take on the onerous task of writing the responses themselves. While we men-tioned in an earlier chapter that a benefit of Net-based interviews is the elimination of the transcribing process, it must be acknowledged that this task is transferred to the participant. Most of us know of us know from our experience communicating with email that when a complex concept requires explanation, it is often easier to simply pick up the phone and communicate the information verbally. As such, when considering using the Net for interviews, it is important to weigh the advantages and disadvantages of telephone and Net-based interviews--especially when the responses might include complex concepts. Alternatively, the advantages of conducting Net-based interviews include the ability to access geographically dispersed respondents more cost effectively than with telephone interviews, the ability to time shift to make the process more con-venient for participant and researcher, and (as already mentioned) the elimination of the transcribing process.\\

\hspace*{0.5cm} Once the e-researcher has made a decision  to use the Net to collect interview data, trust should be built through establishing rapport with the participant. Estab-lishing rapport requires the researcher to be able to ''put him--or herself in the role of the respondents and attempt to see the situation from their perspective, rather than impose the world of academia and preconceptions upon them. Close rapport with respondents opens doors to more informed research'' (Fontana \& Frey, 1994, p. 367), and is discussed further in the next section.\\
\hspace*{0.5cm} Finally, in addition to the automatic transcripts that are created in text-based asynchronous interviews, e-researchers should keep a reflective journal consisting of their speculations, feelings, problems, and prejudices towards the interview process. Keeping a journal helps e-researchers become aware of their biases and assumptions.

\end{document} 